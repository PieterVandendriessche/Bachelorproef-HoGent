%%=============================================================================
%% Inleiding
%%=============================================================================

\chapter{\IfLanguageName{dutch}{Inleiding}{Introduction}}
\label{ch:inleiding}
%%De inleiding moet de lezer net genoeg informatie verschaffen om het onderwerp te begrijpen en in te zien waarom de onderzoeksvraag de moeite waard is om te onderzoeken. In de inleiding ga je literatuurverwijzingen beperken, zodat de tekst vlot leesbaar blijft. Je kan de inleiding verder onderverdelen in secties als dit de tekst verduidelijkt. Zaken die aan bod kunnen komen in de inleiding~\autocite{Pollefliet2011}:

%%\begin{itemize}
 %% \item context, achtergrond
%%  \item afbakenen van het onderwerp
%%  \item verantwoording van het onderwerp, methodologie
 %% \item probleemstelling
 %% \item onderzoeksdoelstelling
 %% \item onderzoeksvraag
  %%\item \ldots
%%\end{itemize}

\section{\IfLanguageName{dutch}{Probleemstelling}{Problem Statement}}
\label{sec:probleemstelling}
Een leven zonder smartphone en zijn applicaties is de dag van vandaag ondenkbaar.  Die mobiele applicaties geven ons toegang tot allerhande informatie, entertainment en is de poort naar onze sociale interactie.

Veel mensen met enige vorm van beperking zijn niet in staat om een smartphone te gebruiken zoals de doorsnee mens en missen dan ook vaak de voordelen van de diverse apps. De essentie van het probleem ligt jammer genoeg vaak bij de ontwikkeling van deze applicaties.  Doordat er in de ontwikkelingsfase te weinig rekening gehouden wordt met deze doelgroep sluiten de makers en de bedrijven niet alleen onbewust een aantal mensen uit maar de slechte user experience kan zelfs lijden tot imagoschade.  

Toch bieden mobiele platformen tal van functionaliteiten aan om een applicatie toegankelijker te maken, maar het toepassen van deze functionaliteiten is voor velen een werkpunt. Toegankelijkheid is uiteraard geen exacte wetenschap en net daardoor is het niet evident om zonder de actieve medewerking van iemand met een beperking te kunnen beoordelen of de gedane inspanningen voldoende zijn. Een duidelijke indicator wanneer een mobiele applicatie voldoende toegankelijk is ontbreekt.

Een Europese richtlijn stelt dat websites en mobiele applicaties van overheden toegankelijk moeten zijn voor iedereen. 
De Vlaamse overheid volgt hierbij deze richtlijn en zal op 23 juni 2021 van kracht zijn ~\autocite{vlaanderenVerplichting}. Dit betekent concreet ook dat externe partijen die mobiele applicaties ontwikkelen voor de overheid deze richtlijn ook zullen moeten volgen.

%Uit je probleemstelling moet duidelijk zijn dat je onderzoek een meerwaarde heeft voor een concrete doelgroep. De doelgroep moet goed gedefinieerd en afgelijnd zijn. Doelgroepen als ``bedrijven,'' ``KMO's,'' systeembeheerders, enz.~zijn nog te vaag. Als je een lijstje kan maken van de personen/organisaties die een meerwaarde zullen vinden in deze bachelorproef (dit is eigenlijk je steekproefkader), dan is dat een indicatie dat de doelgroep goed gedefinieerd is. Dit kan een enkel bedrijf zijn of zelfs één persoon (je co-promotor/opdrachtgever).

\section{\IfLanguageName{dutch}{Onderzoeksvraag}{Research question}}
\label{sec:onderzoeksvraag}



Dit onderzoek zal voor twee mobiele platformen, namelijk iOS en Android, nagaan hoe een ontwikkelaar zijn applicaties toegankelijk kan maken. Daarnaast gaat dit onderzoek ook na of er vertrekkende van bestaande richtlijnen die bedoeld zijn om toegankelijkheid van websites te toetsen, omgezet kunnen worden naar richtlijnen voor mobiele applicaties te toetsen. Aan de hand van die opgestelde richtlijnen zal dan ook de toegankelijkheid van verscheidene mobiele applicaties kunnen getest worden. De uitgevoerde testen in dit onderzoek zullen een beeld geven over hoe  toegankelijk deze mobiele applicaties zijn.

Concreet bestaat dit onderzoek uit de volgende deelonderzoeksvragen:
\begin{itemize}
    \item Wat zijn de overeenkomsten en verschillen tussen de platformen iOS en Android inzake toegankelijkheid?
    \item Hoe kan men verbeteringen aanbrengen in een mobiele applicatie die de toegankelijkheid voor gebruikers met een beperking verbeteren?
\end{itemize}

%Wees zo concreet mogelijk bij het formuleren van je onderzoeksvraag. Een onderzoeksvraag is trouwens iets waar nog niemand op dit moment een antwoord heeft (voor zover je kan nagaan). Het opzoeken van bestaande informatie (bv. ``welke tools bestaan er voor deze toepassing?'') is dus geen onderzoeksvraag. Je kan de onderzoeksvraag verder specifiëren in deelvragen. Bv.~als je onderzoek gaat over performantiemetingen, dan 

\section{\IfLanguageName{dutch}{Onderzoeksdoelstelling}{Research objective}}
\label{sec:onderzoeksdoelstelling}

%aWat is het beoogde resultaat van je bachelorproef? Wat zijn de criteria voor succes? Beschrijf die zo concreet mogelijk. Gaat het bv. om een proof-of-concept, een prototype, een verslag met aanbevelingen, een vergelijkende studie, enz.

Dit onderzoek beoogt het in kaart brengen van richtlijnen waarop ontwikkelaars zich kunnen baseren. Aan de hand van die richtlijnen kan er dan door de ontwikkelaars gemakkelijk nagegaan worden of ze inzake toegankelijkheid op het juiste spoor zitten.

Daarnaast zal er ook concreet nagegaan worden, voor zowel iOS als Android, welke relevante functionaliteiten er beschikbaar zijn.
Dit onderzoek biedt dus een aantal duidelijke richtlijnen om te kunnen aftoetsen hoe toegankelijk de mobile applicatie is  maar ook aanbevelingen wanneer blijkt dat er nog aandacht aan toegankelijkheid besteed moet worden.

\section{\IfLanguageName{dutch}{Opzet van deze bachelorproef}{Structure of this bachelor thesis}}
\label{sec:opzet-bachelorproef}

% Het is gebruikelijk aan het einde van de inleiding een overzicht te
% geven van de opbouw van de rest van de tekst. Deze sectie bevat al een aanzet
% die je kan aanvullen/aanpassen in functie van je eigen tekst.

De rest van deze bachelorproef is als volgt opgebouwd:

In Hoofdstuk~\ref{ch:stand-van-zaken} wordt een overzicht gegeven van de stand van zaken binnen het onderzoeksdomein, op basis van een literatuurstudie.

In Hoofdstuk~\ref{ch:methodologie} wordt de methodologie toegelicht en worden de gebruikte onderzoekstechnieken besproken om een antwoord te kunnen formuleren op de onderzoeksvragen.

% TODO: Vul hier aan voor je eigen hoofstukken, één of twee zinnen per hoofdstuk

In Hoofdstuk~\ref{ch:conclusie}, tenslotte, wordt de conclusie gegeven en een antwoord geformuleerd op de onderzoeksvragen. Daarbij wordt ook een aanzet gegeven voor toekomstig onderzoek binnen dit domein.