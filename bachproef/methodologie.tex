%%=============================================================================
%% Methodologie
%%=============================================================================

\chapter{\IfLanguageName{dutch}{Methodologie}{Methodology}}
\label{ch:methodologie}

%% TODO: Hoe ben je te werk gegaan? Verdeel je onderzoek in grote fasen, en
%% licht in elke fase toe welke stappen je gevolgd hebt. Verantwoord waarom je
%% op deze manier te werk gegaan bent. Je moet kunnen aantonen dat je de best
%% mogelijke manier toegepast hebt om een antwoord te vinden op de
%% onderzoeksvraag.


In dit onderzoek wensen we graag inzicht te krijgen hoe men als een ontwikkelaar de toegankelijkheid van een mobiele applicatie kan verhogen. Ook wenst dit onderzoek een algemeen beeld te geven van de situatie rondom toegankelijkheid van mobiele applicaties. Om deze doelen te behalen werd dit onderzoek in 3 fasen opgesplitst. Elke fase wordt hieronder uitgelegd en komt overeen met de volgende hoofdstukken.

\section{Functionaliteiten in Android en iOS}
\label{section:Functionaliteiten in Android en iOS}
In dit hoofdstuk gaan we specifiek voor Android en iOS na welke functionaliteiten beschikbaar zijn. Dit wordt besproken per domein, zoals deze beschreven staan in \ref{sec:beperkingen}. We bespreken deze functionaliteiten, waarbij we dieper zullen ingaan op hoe een ontwikkelaar de functionaliteit goed kan implementeren in zijn applicatie. Dit is van groot belang, want als een ontwikkelaar niet gebruik maakt van de \gls{API}'s die beschikbaar zijn, zullen bepaalde functionaliteiten van een gebruiker niet bruikbaar zijn in een applicatie.

Naast het diepgaand bespreken van de verschillende functionaliteiten zal er in dit hoofdstuk een overzicht zijn van de verschillende functionaliteiten. Zodat een ontwikkelaar een beter inzicht kan krijgen in het aanbod van beide mobiele platformen en de capaciteiten ervan.
\newpage
%In dit hoofdstuk werd dieper ingegaan op de verschillende functionaliteiten beschikbaar voor beide mobiele platformen. Er werd per domein, zoals deze besproken werden in \ref{sec:beperkingen} onderzocht welke functionaliteiten er beschikbaar zijn, en welke aanpassingen een ontwikkelaar moet doen om deze functionaliteiten bruikbaar te maken in zijn mobiele applicatie. 
%Naast het bespreken van deze functionaliteiten, werd ook een overzicht gemaakt met deze functionaliteiten. Aan de hand van dit overzicht kan een ontwikkelaar een beter inzicht krijgen in de verschillende mobiele platformen en de capaciteiten ervan.
\section{Richtlijnen voor toegankelijkheid mobiele applicaties}
\label{section:Richtlijnen voor toegankelijkheid mobiele applicaties}
Ontwikkelaars hebben nood aan een duidelijke set van richtlijnen voor het toetsen van hun mobiele applicaties op toegankelijkheid. Toch zijn al deze richtlijnen reeds beschikbaar, maar er ontbreekt iets cruciaal. Een maatstaf waarbij ontwikkelaars kunnen toetsen of hun applicatie toegankelijk genoeg is, en voor welk specifiek domein.

Voor een vlot inzicht te kunnen krijgen over de toegankelijkheidssituatie van een applicatie zal dus een maatstaf gecreëerd worden, waarbij we bestaande richtlijnen koppelen aan een gewicht. ...... AANVULLEN TODO AANVULLEN. 

\section{Toetsen toegankelijkheid  a.d.h.v. richtlijnen}
\label{section:Toetsen toegankelijkheid a.d.h.v. richtlijnen}
Lorem ipsum dolor sit amet, consectetur adipiscing elit. Mauris nec risus eget sem rhoncus lobortis non dignissim velit. Nulla auctor turpis eu turpis malesuada auctor non sed dui. Praesent sit amet euismod urna. Aliquam erat volutpat. Aliquam auctor eget ligula in bibendum. Donec ultrices lectus in accumsan dictum. Donec libero risus, tempus in semper ut, venenatis vitae nisl. Vestibulum vel augue libero. Mauris quis tortor id dolor consectetur dignissim. Quisque facilisis massa ut mi ultricies, eget sollicitudin tellus mollis. Aliquam eget est quam. Fusce at pretium est, vitae auctor dui.

Nunc quis quam id elit efficitur blandit eget et nibh. Donec sed scelerisque dolor. Cras ac tortor lacus. Suspendisse imperdiet tincidunt hendrerit. Etiam at dictum ante. Cras eu diam ut leo faucibus laoreet in eu urna. Proin ut mollis est. Maecenas et consequat nisi. Vivamus scelerisque hendrerit augue. Suspendisse tincidunt rutrum tortor ut molestie. Pellentesque habitant morbi tristique senectus et netus et malesuada fames ac turpis egestas. Aliquam erat volutpat. Curabitur bibendum nisl est. Aliquam tincidunt diam eget nibh auctor, ac egestas ex facilisis. Donec felis sapien, dictum ac velit et, facilisis tempus mauris. Proin mauris diam, dignissim nec metus vitae, condimentum volutpat odio.

