\chapter{\IfLanguageName{dutch}{Richtlijnen voor mobiele applicaties}{Richtlijnen voor  mobiele applicaties}}
\label{ch:Richtlijnen voor toegankelijkheid mobiele applicaties}
Een ontwikkelaar moet kunnen nagaan of zijn mobiele applicatie toegankelijk is. De bestaande richtlijnen zijn intimiderend. En men kan moeilijk toetsen wanneer men voldaan heeft. In dit hoofdstuk wordt aan de hand van de WCAG richtlijnen een maatstaf opgesteld voor ontwikkelaars. Hierdoor kan vlot een inzicht gecreëerd worden over de toegankelijkheid van een mobiele applicatie.

\section{WCAG richtlijnen}
\label{sec:WCAGrichtlijn}

Zoals reeds besproken is in sectie \ref{sec:wetgeving} vormen de WCAG richtlijnen de basis voor dit onderzoek. Meer bepaald de richtlijnen die besproken zijn in WCAG 2.1. Deze richtlijnen bevatten extra criteria gericht op het gebruik van touchscreens, visuele beperkingen en cognitieve beperkingen.
De succescriteria beschreven in de WCAG richtlijnen zijn technologie onafhankelijk opgesteld. Dit wil zeggen dat ze zowel op mobiele platformen als computers kunnen toegepast worden. Vaak wordt het woord 'web' gebruikt in een richtlijn, wat in de meeste gevallen vervangen kan worden door 'mobiele applicatie'.
 \autocite{w3cTechnologyNeutral}

De WCAG 2.1 richtlijnen zijn opgebouwd in een specifieke structuur. De richtlijnen en de daarbij horende succescriteria zijn onderverdeeld onder vier verschillende principes. Aan deze principes moet voldaan worden om een toegankelijke applicatie te hebben. Deze principes zijn respectievelijk: 
\begin{itemize}
    \item Waarneembaar: gebruikers moeten de informatie (op het scherm) kunnen waarnemen.
        \item Bedienbaar: gebruikers moeten kunnen de elementen kunnen bedienen, waaronder ook navigatie.
        \item Begrijpelijk: gebruikers moeten de informatie en bediening van een applicatie verstaan.
        \item Robuust: gebruikers moeten de inhoud blijven kunnen gebruiker wanneer ze nieuwe technologie gebruiken.
\end{itemize}

Binnen de principes zijn er richtlijnen. Deze richtlijnen bevatten de doelen die men moet behalen voor aan een principe te voldoen. Die doelen kunnen behaald worden door de applicatie te toetsen aan succescriteria. Een succescriteria schrijft vereisten voor waaraan voldaan moet worden om een richtlijn te kunnen behalen. 

Elk succescriteria bevat een niveau, deze niveaus komen overeen met de afstemming op de behoeften van gebruikers met een beperking. Deze levels zijn:
\begin{itemize}
    \item Niveau A: Minimum level, voldoet aan niveau A succescriteria.
    \item Niveau AA: Voldoet aan niveau A succescriteria en niveau AA succescriteria.
    \item Niveau AAA: Voldoet aan niveau A, AA en AAA succescriteria.
\end{itemize}

Binnen dit onderzoek zullen wij ons focussen op niveau A en niveau AA succesfactoren. De WCAG richtlijnen raden af om alle succescriteria te proberen afstemmen tot niveau AAA. Ze vermelden dat het onmogelijk is om deze allemaal te behalen. 

Als voorbeeld:  \emph{Succesfactor 1.2.6: Er wordt een gebarentaalvertolking geleverd voor alle vooraf opgenomen audiocontent in gesyncroniseerde media. (Level AAA)}. Er kan gesteld worden, dat dit onmogelijk is voor vele applicaties om daaraan te voldoen \autocite{WCAG2.1Criteria}.

In het vervolg van deze sectie gaan wordt per principe de richtlijnen en succesfactoren bespreken. Om een indruk te kunnen geven van de potentiële impact op de gebruikers van een succesfactor wordt er een score aan toegekend. Deze score drukt de potentiële impact uit op de gebruikservaring van de app (of een specifieke functionaliteit ervan) voor een van de 4 (ruim gedefinieerde) doelgroepen.
Hoe hoger de score, hoe aannemelijker het is dat een correcte toepassing van dit WCAG-criterium een bovengemiddelde impact heeft op de gebruikerservaring. De scores hebben de volgende betekenis:
\begin{enumerate}
    \item Geen bovengemiddelde impact
    \item Bovengemiddelde impact
    \item Essentiële impact
\end{enumerate}

\newpage
\subsection{Principe 1: Waarneembaar}
\label{sec:waarneembaarWCAG}
Dit principe stelt dat informatie en elementen van een applicatie gepresenteerd moeten worden zodanig dat men die kan waarnemen.


\subsubsection{Richtlijn 1.1 - Tekst alternatieven}
Gebruikers kunnen nood hebben aan een alternatief voor inhoud die geen tekst bevat. Voor deze richtlijn te faciliteren moet aan de succesfactoren in tabel xx voldaan worden.
\begin{table}[H]
    \centering
 \hspace*{-1cm}\begin{tabular}{|l|p{12cm}|} 
        \hline
        \textbf{Succesfactor}                & 1.1.1                                                                                                                                                                                                                                                                                                             \\ 
        \hline
        \textbf{Level}                       & A                                                                                                                                                                                                                                                                                                                                                                             \\ 
        \hline
        \textbf{Naam}                        & Niet-tekst inhoud~                                                                                                                                                                                                                                                                                                                                                            \\ 
        \hline
        \textbf{Slagen van succesfactor}     & \begin{itemize}
            \item Tekstalternatieven voorzien voor niet-tekst inhoud (bv. afbeeldingen, knoppen, …)
        \end{itemize}                                                                                                                                                                                                      \\ 
        \hline
        \textbf{Beschrijving}                & Vooral gebruikers met een visuele beperking zullen gebruik maken van VoiceOver of TalkBack. Ze besturen hun smartphone via audio feedback. Alle elementen die tekst bevatten worden voorgelezen. Sommige elementen bevatten geen of weinig betekenisvolle tekst. Men moet dus goede tekst alternatieven voorzien voor niet-tekst inhoud.  \\ 
        \hline
        \textbf{Impact op gebruikers}        & 
        \begin{itemize}
            \item Visueel: 3, verhoogt het waarnemen van inhoud.
            \item Cognitief: 2, maakt makkelijker om inhoud op te nemen.             
        \end{itemize}                                                                                                                   \\ 
        \hline
        \textbf{Platform specifieke feature} & \begin{itemize}
            \item iOS: VoiceOver, zie \ref{subsec:VoiceOver}.
            \item Android: TalkBack, zie \ref{subsec:TalkBack}
        \end{itemize}                                                                                                                                                                       \\ 
        \hline
        \textbf{Testen}                      & Deze succesfactor kan getest worden door een ontwikkelaar met ofwel TalkBack of VoiceOver de mobiele applicatie te besturen. Wanneer een element niet duidelijk benoemd is (voornamelijk afbeeldingen), kan gesteld worden dat er niet voldaan wordt aan het succescriteria.                                                                                                                                                                                                                        \\
        \hline
    \end{tabular}
\end{table}



%\paragraph{Succesfactor 1.1.1: Niet-tekst inhoud (A)}
%Vooral gebruikers met een visuele beperking zullen gebruik maken van VoiceOver of TalkBack. Ze besturen hun smartphone via audio feedback. Alle elementen die tekst bevatten worden voorgelezen. Sommige elementen bevatten geen of weinig betekenisvolle tekst. Men moet dus goede tekst alternatieven voorzien voor niet-tekst inhoud. Zie sectie \ref{subsec:TalkBack} en \ref{subsec:VoiceOver} voor details over de implementatie. Om te voldoen aan deze succesfactor moet een men: 
%\begin{itemize}
  %  \item Tekstalternatieven voorzien voor niet-tekst inhoud (bv. afbeeldingen, knoppen, ...)
%\end{itemize}
%Een succesvolle implementatie van de succesfactor biedt voordeel aan:
%\begin{itemize}
%    \item Visuele beperking: moeilijkheden met waarnemen beeld.
%    \item Cognitieve beperking: moeilijkheden met opnemen betekenis van elementen (bv. foto's)
%\end{itemize}
%Deze succesfactor kan getest worden dooreen ontwikkelaar met ofwel TalkBack of VoiceOver de mobiele applicatie te besturen. Wanneer een element niet duidelijk benoemd is (voornamelijk afbeeldingen), kan gesteld worden dat er niet voldaan wordt aan het succescriteria.
\subsubsection{Richtlijn 1.2: Tijds-gebaseerde media}
Gebruikers moeten een alternatief hebben voor tijds-gebaseerde media. Dit is zowel vooraf opgenomen media als live media. 
\newpage
\begin{table}[H]
    \centering
    \hspace*{-1cm}\begin{tabular}{|l|p{12cm}|} 
        \hline
        \textbf{Succesfactor}                & 1.2.1                                                                                                                                                                                                                                                                                                             \\ 
        \hline
        \textbf{Level}                       & A                                                                                                                                                                                                                                                                                                                                                                             \\ 
        \hline
        \textbf{Naam}                        & Vooraf opgenomen geluid of beeld~                                                                                                                                                                                                                                                                                                                                                            \\ 
        \hline
        \textbf{Slagen van succesfactor}     & \begin{itemize}
            \item Bij enkel audio: een tekst transcript voorzien.
            \item Bij enkel video: een audio fragment of tekst transcript voorzien.
        \end{itemize}                                                                                                                                                                                                      \\ 
     \hline
    \textbf{Uitzonderingen}     & \begin{itemize}
        \item Wanneer de media een alternatief is voor tekst (die aanwezig is), hoeft er geen rekening gehouden te worden voor dat media element.
    \end{itemize}                                                                                                                                                                                                      \\ 
        \hline
        \textbf{Beschrijving}                & Wanneer men te maken heeft met inhoud die ofwel geluid of beeld moet een alternatief voorzien worden. Bij bijvoorbeeld een audio is een transcriptie die beschrijft wat er gebeurt gewenst. Bij video is een transcriptie of een audio track die beschrijft wat er te horen valt gewenst. \\ 
        \hline
        \textbf{Impact op gebruikers}        & 
        \begin{itemize}
            \item Visueel: 3, verhoogt het waarnemen van inhoud (bij beeld).
            \item Auditief 3, verhoogt waarnemen van inhoud (bij geluid).             
        \end{itemize}                                                                                                                   \\ 
        \hline
        \textbf{Testen}                      & Men kan nagaan of voldaan wordt aan de richtlijn, wanneer men er niet in slaagt, kan gesteld worden dat men niet voldoet aan de richtlijn.                                                                                                                                                                                                            \\
        \hline
    \end{tabular}
\end{table}
\begin{table}[H]
    \centering
    \hspace*{-1cm}\begin{tabular}{|l|p{12cm}|} 
        \hline
        \textbf{Succesfactor}                & 1.2.2                                                                                                                                                                                                                                                                                                             \\ 
        \hline
        \textbf{Level}                       & A                                                                                                                                                                                                                                                                                                                                                                             \\ 
        \hline
        \textbf{Naam}                        & Ondertitelingen bij vooraf opgenomen video’s met audio~                                                                                                                                                                                                                                                                                                                                                            \\ 
        \hline
        \textbf{Slagen van succesfactor}     & \begin{itemize}
            \item Ondertitelingen toevoegen aan video’s waar geluid is.
        \end{itemize}                                                                                                                                                                                                      \\ 
        \hline
        \textbf{Uitzonderingen}     & \begin{itemize}
            \item Wanneer de video een alternatief is voor tekst (die aanwezig is), hoeft er geen rekening gehouden te worden met de succesfactor.
        \end{itemize}                                                                                                                                                                                                      \\ 
        \hline
        \textbf{Beschrijving}                & Ondertitelingen worden vaak gebruikt bij gebruikers met een auditieve beperking. Het biedt een tekstalternatief voor de audio die afgespeeld wordt in een video. \\ 
        \hline
        \textbf{Impact op gebruikers}        & 
        \begin{itemize}
            \item Auditief 3, verhoogt waarnemen van inhoud.             
        \end{itemize}                                                                                                                   \\ 
      \hline
    \textbf{Platform specifieke feature} & \begin{itemize}
        \item iOS: zie \ref{subsec:ondertiteliOS}.
        \item Android: zie \ref{subsec:ondertitelAndroid}
    \end{itemize}                                                                                                                                                                       \\ 
        \hline
        \textbf{Testen}                      & Men kan nagaan of voldaan wordt aan de richtlijn. Wanneer een video, die niet dient als alternatief voor een tekst, geen ondertitelingen ondersteunt, kan gesteld worden dat er niet voldaan wordt.                                                                                                                                                                                              \\
        \hline
    \end{tabular}
\end{table}





%\paragraph{Succesfactor 1.2.1: Vooraf opgenomen geluid of beeld (A)}
%Wanneer men te maken heeft met inhoud die ofwel geluid of beeld moet een alternatief voorzien worden. Bij bijvoorbeeld een audio is een transcriptie die beschrijft wat er gebeurt gewenst. Bij video is een transcriptie of een audio track die beschrijft wat er te horen valt gewenst. Om te voldoen aan deze succesfactor moet men: \begin{itemize}
   % \item Bij enkel audio: een tekst transcript voorzien.
    %\item Bij enkel video: een audio fragment of tekst transcript voorzien.
%\end{itemize}
%Een succesvolle implementatie van de succesfactor biedt voordeel aan:
%\begin{itemize}
    %\item Visuele beperking: moeilijkheden met waarnemen beeld.
  %  \item Auditieve beperking: moeilijkheden waarnemen van geluiden.
%\end{itemize}
%Wanneer de media een alternatief is, bijvoorbeeld voor visuele representatie van een tekst, hoeft geen rekening gehouden te worden met de succesfactor. 
\begin{table}[H]
    \centering
    \hspace*{-1cm}\begin{tabular}{|l|p{12cm}|} 
        \hline
        \textbf{Succesfactor}                & 1.2.3                                                                                                                                                                                                                                                                                                             \\ 
        \hline
        \textbf{Level}                       & A                                                                                                                                                                                                                                                                                                                                                                             \\ 
        \hline
        \textbf{Naam}                        & Audiodescriptie of alternatief bij vooraf opgenomen video’s met audio~                                                                                                                                                                                                                                                                                                                                                            \\ 
        \hline
        \textbf{Slagen van succesfactor}     & \begin{itemize}
            \item Een audio beschrijving te voorzien voor video’s, of
            \item een volledig transcript van de video voorzien.
        \end{itemize}                                                                                                                                                                                                      \\ 
        \hline
        \textbf{Uitzonderingen}     & \begin{itemize}
            \item Wanneer de video een alternatief is voor tekst (die aanwezig is), hoeft er geen rekening gehouden te worden met de succesfactor.
        \end{itemize}                                                                                                                                                                                                      \\ 
        \hline
        \textbf{Beschrijving}                & Gebruikers met een visuele een beperking wensen graag een beschrijving van wat er gebeurt in een video. Een audiodescriptie geeft deze gebruikers via geluid uitleg wat er te zien valt in de video. Dit kan gebeuren via een audio descriptie, of een transcript te voorzien. Dit transcript kan door de screenreader dan voorgelezen worden.\\ 
        \hline
        \textbf{Impact op gebruikers}        & 
        \begin{itemize}
            \item Visueel 3, verhoogt waarnemen van inhoud.    
            \item Cognitief 2, verhoogt verstaanbaarheid bewegende beelden         
        \end{itemize}                                                                                                                   \\ 
      
        \hline
        \textbf{Testen}                      & Men kan nagaan of voldaan wordt aan de richtlijn. Wanneer een video, die niet dient als alternatief voor een tekst, een audio beschrijving heeft, of een transcript van de audio voorzien is. Dan slaagt men met het implementeren van deze succesfactor.                                                                                                                               \\
        \hline
    \end{tabular}
\end{table}




%\paragraph{Succesfactor 1.2.2: Ondertitelingen bij vooraf opgenomen video's met audio (A)}
%Ondertitelingen worden vaak gebruikt bij gebruikers met een auditieve beperking. Het biedt een tekstalternatief voor de audio die afgespeeld wordt in een video. Om te voldoen aan deze succesfactor dient men: 
%\begin{itemize}
%    \item Ondertitelingen toe te voegen aan video's waar geluid is.
%\end{itemize}
%Ook bij deze succesfactor geldt dat wanneer de video een alternatief is voor reeds bestaande tekst, men geen rekening hoeft te houden ermee.

%Een succesvolle implementatie van de succesfactor biedt voordeel aan:
%\begin{itemize}
%    \item Auditieve beperking: moeilijkheden waarnemen van geluiden.
%\end{itemize}
%Zie sectie \ref{subsec:ondertitelAndroid} en \ref{subsec:ondertiteliOS} voor de implementatie/functionaliteit van ondertitelingen in Android en iOS. 
%\paragraph{Succesfactor 1.2.3: Audiodescriptie of alternatief bij vooraf opgenomen video's met audio (A)}
%Gebruikers met een visuele een beperking wensen graag een beschrijving van wat er gebeurt in een video. Een audiodescriptie geeft deze gebruikers via geluid uitleg wat er te zien valt in de video.
%Dit kan gebeuren via een audio descriptie, of een transcript te voorzien. Dit transcript kan door de screenreader dan voorgelezen worden.
%Om te voldoen aan deze succesfactor dient men: 
%\begin{itemize}
%%    \item Een audio beschrijving te voorzien voor video's, of
 %      \item een transcript van de audio voorzien voor video's te voorzien.
    
%\end{itemize}
%Men hoeft geen rekening te houden met deze succesfactor wanneer de video een %alternatief is voor een tekst.

%Een succesvolle implementatie van de succesfactor biedt voordeel aan:
%\begin{itemize}
%    \item Visuele beperking: moeilijkheden waarnemen van beeld.
%        \item Cognitieve beperking: moeilijkheden verstaan van bewegende beelden.
%\end{itemize}

\begin{table}[H]
    \centering
    \hspace*{-1cm}\begin{tabular}{|l|p{12cm}|} 
        \hline
        \textbf{Succesfactor}                & 1.2.4                                                                                                                                                                                                                                                                                                             \\ 
        \hline
        \textbf{Level}                       & AA                                                                                                                                                                                                                                                                                                                                                                             \\ 
        \hline
        \textbf{Naam}                        & Ondertitelingen bij live video’s met audio~                                                                                                                                                                                                                                                                                                                                                            \\ 
        \hline
        \textbf{Slagen van succesfactor}     & \begin{itemize}
            \item Ondertitelingen toevoegen aan live video’s waar geluid is.
        \end{itemize}                                                                                                                                                                                                      \\ 
        \hline
        \textbf{Beschrijving}                & Gebruikers met een auditieve beperking hebben vooral baat bij het gebruik van ondertite- lingen in een live video met geluid.  \\ 
        \hline
        \textbf{Impact op gebruikers}        & 
        \begin{itemize}
            \item Auditief 3, verhoogt waarnemen van inhoud.             
        \end{itemize}                                                                                                                   \\ 
        \hline
        \textbf{Platform specifieke feature} & \begin{itemize}
            \item iOS: zie \ref{subsec:ondertiteliOS}.
            \item Android: zie \ref{subsec:ondertitelAndroid}
        \end{itemize}                                                                                                                                                                       \\ 
        \hline
        \textbf{Testen}                      & Elke live video met geluid moet ondertitelingen bevatten bij het activeren ervan.                                                                                                                                                                                    \\
        \hline
    \end{tabular}
\end{table}

\paragraph{Succesfactor 1.2.4: Ondertitelingen bij live video's met audio (AA)}
Gebruikers met een auditieve beperking hebben vooral baat bij het gebruik van ondertitelingen in een live video met geluid. Om te voldoen aan deze succesfactor moet men: 
\begin{itemize}
    \item Ondertitelingen toevoegen aan live video's waar geluid is.
\end{itemize}
Een succesvolle implementatie van de succesfactor biedt voordeel aan:
\begin{itemize}
    \item Auditieve beperking: moeilijkheden waarnemen van geluiden.
\end{itemize}

Zie sectie \ref{subsec:ondertitelAndroid} en \ref{subsec:ondertiteliOS} voor de implementatie/functionaliteit van ondertitelingen in Android en iOS. 

\paragraph{Succesfactor 1.2.5: Audiodescriptie bij vooraf opgenomen video's met audio (AA)}
Deze succesfactor leunt sterk aan tegen succesfactor 1.2.3. Deze overlapping komt doordat men in 1.2.5 (niveau AA) iets hogere vereisten stelt. 
In 1.2.3 heeft men de keuze voor een transcript, of een audiodescriptie te voorzien. In 1.2.5 wordt men verplicht om een audiodescriptie te voorzien. 

Om te voldoen aan deze succesfactor dient men: 
\begin{itemize}
    \item Een audio beschrijving te voorzien voor video's.
    
\end{itemize}
Men hoeft geen rekening te houden met deze succesfactor wanneer de video een alternatief is voor een tekst.

Een succesvolle implementatie van de succesfactor biedt voordeel aan:
\begin{itemize}
    \item Visuele beperking: moeilijkheden waarnemen van beeld.
    \item Cognitieve beperking: moeilijkheden verstaan van bewegende beelden.
\end{itemize}

\subsubsection{Richtlijn 1.3 - Aanpasbaar}
Deze richtlijn focust op het feit dat alle informatie beschikbaar is in de vorm waarin hij kan waargenomen worden. Bijvoorbeeld door het gebruik van VoiceOver of TalkBack. Daarvoor moet de informatie in een applicatie kunnen waargenomen worden door die software.
\paragraph{Succesfactor 1.3.1:  Informatie en relaties tussen informatie (A)}
Een gebruiker die gebruik maakt van een screenreader voor te navigeren. Alle informatie op het scherm moet daarvoor kunnen bepaald worden door die software. Voor deze succesfactor te implementeren dient:
\begin{itemize}
    \item Alle informatie waarneembaar te zijn door een screenreader.
\end{itemize}
Een succesvolle implementatie van de succesfactor biedt voordeel aan:
\begin{itemize}
    \item Visuele beperking: elementen op scherm worden correct voorgelezen.
\end{itemize}
Zie sectie \ref{subsec:TalkBack} en \ref{subsec:VoiceOver} voor details over de implementatie.

\paragraph{Succesfactor 1.3.2:  Betekenisvolle volgorde (A)}
Bij het gebruik van een screenreader kan volgorde van de informatie vaak van belang zijn. Een gebruiker met een visuele beperking, voornamelijk blinden zullen moeite hebben met de context van de informatie. Wanneer deze niet logisch gerangschikt, gegroepeerd staat, kan dit problemen geven.
Om deze succesfactor succesvol te implementeren dient:
\begin{itemize}
    \item De inhoud logisch geordend zijn.
    \item Gerelateerde informatie gegroepeerd staan.
\end{itemize}
Een succesvolle implementatie van de succesfactor biedt voordeel aan:
\begin{itemize}
    \item Visuele beperking: elementen op scherm worden correct voorgelezen.
\end{itemize}
Zie sectie \ref{subsec:TalkBack} en \ref{subsec:VoiceOver} voor details over hoe men inhoud logisch kan groeperen binnen een applicatie.
\paragraph{Succesfactor 1.3.3:  Zintuiglijke eigenschappen (A)}
Deze succesfactor verhindert dat gebruikers met een visuele beperking een mobiele applicatie niet kunnen bedienen door onduidelijke instructies. Bijvoorbeeld: een blinde gebruiker zal niks zijn met de instructie om op de derde knop links te drukken. Voor zo'n actie uit te voeren mag men niet blind zijn. 

Hetzelfde voor mensen met een auditieve beperking, wanneer instructies enkel via geluid worden gegeven, kan deze gebruiker geen actie ondernemen.

Om deze succesfactor succesvol te implementeren dient men: 
\begin{itemize}
    \item Instructies te geven die waarneembaar zijn voor verschillende zintuigen (visueel, auditief, ...)
\end{itemize}

Men kan in iOS eventueel inspelen als developer op de properties in de klasse \emph{UIAccessibility}. In Android kan men van bepaalde instellingen de waarden opvragen. 
Een succesvolle implementatie van de succesfactor biedt voordeel aan mensen met een:
\begin{itemize}
    \item Visuele beperking
    \item Auditieve beperking
\end{itemize}

\paragraph{Succesfactor 1.3.4:  Oriëntatie scherm (AA)}
Een smartphone kan gebruikt worden in portret- of landschapmodus. Men kan wisselen van modus door het apparaat te roteren. 
Gebruikers met een motorische beperking zullen vaak niet de mogelijkheid hebben om te wisselen van modus. Daarom stelt deze succesfactor dat de inhoud van een mobiele applicatie hetzelfde moet zijn in zowel portret- als landschapmodus.

Om te voldoen aan deze succesfactor dient:
\begin{itemize}
    \item Alle informatie en functionaliteiten identiek zijn in zowel portet- als landschapmodus.
\end{itemize}

Een ontwikkelaar moet rekening houden met de mogelijkheid tot roteren van het scherm bij het ontwikkelen van de applicatie. Wanneer men de lay-out opmaakt dient men hier aandacht aan te besteden.

Een succesvolle implementatie van de succesfactor biedt voordeel aan mensen met een motorische beperking.

\paragraph{Succesfactor 1.3.5:  Identificeerbaar doel van invoerveld (AA)}
Wanneer men een formulier of veld moet invullen moet elk veld een duidelijk identificeerbaar doel hebben. Dit kan zijn door het gebruik van labels, maar ook door de juiste metadata aan een veld te geven.
Wanneer metadata toegevoegd is, kan het besturingssysteem eventueel aanbevelingen geven aan de gebruiker welke informatie ingevuld moet worden.

Om te voldoen aan deze succesfactor dient men:
\begin{itemize}
    \item Aan elk veld een duidelijk doel te koppelen.
\end{itemize}

Voor aan deze succesfactor te voldoen kan men dus kiezen voor duidelijke labels, en eventueel metadata te koppelen aan velden. Binnen iOS kan men gebruik maken van \emph{Text input traits}, in Android kan men een \emph{inputType} koppelen aan een element.

Een succesvolle implementatie van de succesfactor biedt voordeel aan mensen met een:
\begin{itemize}
    \item Cognitieve beperking: moeilijkheden met taal, of geheugen
    \item Motorische beperking: moeite met invullen van velden
    \item Visuele beperking: duidelijkere beschrijving verwachte input (screenreader)
\end{itemize}

\subsubsection{Richtlijn 1.4 - Onderscheidbaar}
https://www.wuhcag.com/images-of-text/
\paragraph{Succesfactor 1.4.1:  Gebruik van kleur (A)}
\paragraph{Succesfactor 1.4.2:  Controle over geluid (A)}
\paragraph{Succesfactor 1.4.3:  Contrast (Minimum) (AA)}
\paragraph{Succesfactor 1.4.4:  Schalen van tekst (AA)}
\paragraph{Succesfactor 1.4.5:  Tekst in een afbeelding (AA)}
\paragraph{Succesfactor 1.4.10:  ENGLISG REFLOW (AA)}
\paragraph{Succesfactor 1.4.11:  Niet-tekst contrast (AA)}
\paragraph{Succesfactor 1.4.12:  Tekst regelafstand (AA)}
\paragraph{Succesfactor 1.4.13:  Inhoud bij focussen op element (AA)}

\subsection{Principe 2: Bedienbaar}
\label{sec:bedienbaarWCAG}
\subsubsection{Richtlijn 2.1 - Toetsenbord toegankelijk}
\paragraph{Succesfactor 2.1.1:  Toetsenbord (A)}
\paragraph{Succesfactor 2.1.2:  Geen toetsenbordval (A)}
\subsubsection{Richtlijn 2.2 - Genoeg tijd}
