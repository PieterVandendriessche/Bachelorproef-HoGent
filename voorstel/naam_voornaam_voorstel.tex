%==============================================================================
% Sjabloon onderzoeksvoorstel bachelorproef
%==============================================================================
% Gebaseerd op LaTeX-sjabloon ‘Stylish Article’ (zie voorstel.cls)
% Auteur: Jens Buysse, Bert Van Vreckem

\documentclass[fleqn,10pt]{voorstel}



%------------------------------------------------------------------------------
% Metadata over het voorstel
%------------------------------------------------------------------------------

\JournalInfo{HoGent Bedrijf en Organisatie}
\Archive{Bachelorproef 2018 - 2019} % Of: Onderzoekstechnieken

%---------- Titel & auteur ----------------------------------------------------

% TODO: geef werktitel van je eigen voorstel op
\PaperTitle{Toegankelijkheid native apps in Android en iOS}
\PaperType{Onderzoeksvoorstel Bachelorproef} % Type document

% TODO: vul je eigen naam in als auteur, geef ook je emailadres mee!
\Authors{Pieter Vandendriessche\textsuperscript{1}} % Authors
\CoPromotor{Roel Van Gils\textsuperscript{2} (Eleven Ways)}
\affiliation{\textbf{Contact:}
  \textsuperscript{1} \href{mailto:pieter.vandendriessche.y0257@student.hogent.be}{pieter.vandendriessche.y0257@student.hogent.be};
  \textsuperscript{2} \href{mailto:roel@elevenways.be}{roel@elevenways.be};
}

%---------- Abstract ----------------------------------------------------------

\Abstract{Toegankelijkheid bij mobiele apps laat een ontwikkelaar toe om zoveel mogelijk gebruikers te bereiken. Zonder de nodige aanpassingen dreigt de groep gebruikers met een beperking geen toegang meer te hebben tot informatie, sociale interactie en economische betrokkenheid. Onwetendheid zorgt ervoor dat de groot aantal functionaliteiten aangeboden in zowel iOS als Android niet gebruikt worden voor apps toegankelijker te maken. In dit onderzoek wordt nagegaan in welke mate beide platformen verschillen inzake toegankelijkheid. Dit onderzoek zal enkele richtlijnen bepalen die de ontwikkelaar moet toepassen om zijn applicatie toegankelijk te maken voor zoveel mogelijk gebruikers. Aan de hand van deze richtlijnen zal er ook onderzocht worden of bepaalde populaire applicaties toegankelijk genoeg zijn.
Ook wordt er gekeken of verbeteringen voor toegankelijkheid een positief effect heeft op het gebruikersgemak van apps. Er wordt verwacht van dit onderzoek dat er meer aandacht moet gespendeerd worden aan toegankelijkheid voor mobiele apps. Hopelijk mag dit onderzoek bijdragen aan toekomstige integratie van toegankelijkheid in mobiele apps.  }

%---------- Onderzoeksdomein en sleutelwoorden --------------------------------
% TODO: Sleutelwoorden:
%
% Het eerste sleutelwoord beschrijft het onderzoeksdomein. Je kan kiezen uit
% deze lijst:
%
% - Mobiele applicatieontwikkeling
% - Webapplicatieontwikkeling
% - Applicatieontwikkeling (andere)
% - Systeembeheer
% - Netwerkbeheer
% - Mainframe
% - E-business
% - Databanken en big data
% - Machineleertechnieken en kunstmatige intelligentie
% - Andere (specifieer)
%
% De andere sleutelwoorden zijn vrij te kiezen

\Keywords{Onderzoeksdomein. Mobiele applicatieontwikkeling --- Toegankelijkheid } % Keywords
\newcommand{\keywordname}{Sleutelwoorden} % Defines the keywords heading name

%---------- Titel, inhoud -----------------------------------------------------

\begin{document}

\flushbottom % Makes all text pages the same height
\maketitle % Print the title and abstract box
\tableofcontents % Print the contents section
\thispagestyle{empty} % Removes page numbering from the first page

%------------------------------------------------------------------------------
% Hoofdtekst
%------------------------------------------------------------------------------

% De hoofdtekst van het voorstel zit in een apart bestand, zodat het makkelijk
% kan opgenomen worden in de bijlagen van de bachelorproef zelf.
%---------- Inleiding ---------------------------------------------------------

\section{Introductie} % The \section*{} command stops section numbering
\label{sec:introductie}

Smartphones hebben een grote impact op ons hedendaagse leven. Het heeft de sociale en economische betrokkenheid van een individu veranderd.  Doch is het gebruik van een smartphone gelimiteerd voor mensen met een beperking
~\autocite{morris2014wireless}. Deze groep dreigt dan ook door deze limitatie uitgesloten te worden van bepaalde informatie. De Belgische overheid heeft doelstellingen gemaakt om tegen 22 juni 2021 al hun mobiele applicaties toegankelijk te maken voor mensen met een beperking. Hierbij volgen ze de richtlijnen op van de Europese Unie voor toegankelijkheid van digitale informatie~\autocite{Knacktoegankelijkheid2018}.

Tegenwoordig is toegankelijkheid een issue die we niet kunnen en mogen vermijden. Toch is er binnen de software-development een gebrek aan duidelijke richtlijnen. Juist daarom wordt dit onderwerp vaak overgeslagen bij het maken van mobiele applicaties. Ook de exponentiële groei van innovatie bij smartphones beperkt het opleggen van duidelijke richtlijnen~\autocite{diaz2014accessibility}. 
\\~\\
Er zijn  tal van functionaliteiten die beschikbaar worden gesteld op mobiele platformen om deze toegankelijker te maken. Ontwikkelaars worden daarbij voorzien van uitgebreide software bibliotheken. Deze functionaliteiten en bibliotheken verschillen per platform, waardoor het aanbod aan voorzieningen voor bepaalde beperkingen kan verschillen. In dit onderzoek gaan we voor de platformen Android en iOS het volgende nagaan:


% TODO: Deftige formulering van onderzoeksvragen


\begin{itemize}
    \setlength\itemsep{0.5 em}
    \item Wat zijn de overeenkomsten en verschillen tussen de platformen iOS en Android inzake toegankelijkheid?
  \item Hoe kan men verbeteringen aanbrengen in een mobiele applicatie die de toegankelijkheid voor gebruikers met een beperking verbeteren?
  \item Hebben toegankelijkheid verbeteringen een positief effect op het algemene gebruikersgemak van apps? 
  
\end{itemize}

%---------- Stand van zaken ---------------------------------------------------

\section{State-of-the-art}
\label{sec:state-of-the-art}

Onderzoek naar toegankelijkheid in mobiele applicaties heeft er in het verleden al veel plaatsgevonden. Door de snelle innovatie en groei van mobiele platformen lijken vele onderzoeken gedateerd. Vele nieuwe functionaliteiten ontbreken in deze onderzoeken.
Vaak zijn de onderzoeken ook voor een specifieke beperking. Zo omschreef het onderzoek van \citeauthor{leporini2012interacting} wat de impact van de VoiceOver functie in iOS was op mensen met een visuele beperking. Er werd in dit onderzoek geconcludeerd dat ondanks de krachtige mogelijkheden van VoiceOver de applicaties niet voldoende de interactieve elementen beschrijven zodat VoiceOver deze beschrijving kan voorlezen~\autocite{leporini2012interacting}. 

Het onderzoek van \citeauthor{diaz2014accessibility} focust zich op de oudste generatie, vaak hebben zij te kampen met een lichamelijke achteruitgang. Ook deze groep heeft daardoor nood aan toegankelijkheid in mobiele applicaties. Gedurende dit onderzoek heeft men onderzoek gedaan om duidelijke richtlijnen te maken voor toegankelijkheid bij ouderen. Ze concluderen dat elke applicatie toegankelijk zou moeten zijn om sociale uitsluiting te voorkomen~\autocite{diaz2014accessibility}. 

Een vergelijkende studie tussen Android en iOS over de beschikbare functionaliteiten voor gebruikers werd door \citeauthor{10.1007/978-3-319-07638-6_14} uitgevoerd. Deze studie heeft per type beperking een opsomming gemaakt van enkele functionaliteiten die er per platform beschikbaar zijn voor de gebruiker. De onderzoeker vermelde dat er in de toekomst nieuwe functionaliteiten zullen zijn die nog betere toegankelijkheid zal verzorgen voor mensen met een beperking bij het gebruik van mobiele applicaties~\autocite{10.1007/978-3-319-07638-6_14}. 
\\~\\
Dit onderzoek zal verder gaan dan enkel de functionaliteiten voor gebruikers, ook de functionaliteiten die beschikbaar gesteld worden voor de ontwikkelaar om zijn app meer toegankelijk te maken voor zowel iOS en Android worden gekaderd. Ook wordt de focus op het aanbrengen van verbeteringen voor verschillende soorten beperkingen gelegd. Daarnaast zal er nagegaan worden of toegankelijkheid verbeteringen een positief effect zou kunnen hebben op het algemene gebruikersgemak van de gebruiker die geen of zich niet identificeert met een beperking.


%---------- Methodologie ------------------------------------------------------
\section{Methodologie}
\label{sec:methodologie}

Voor de onderzoeksvragen beantwoord worden is er nood aan inzicht in de verschillende beperkingen die aan bod zullen komen gedurende dit onderzoek. 

Wanneer we inzicht hebben gevormd in de verschillende beperkingen zal er voor de eerste onderzoeksvraag een literatuurstudie plaatsvinden. Daarin zullen we per beperking de functionaliteiten voor zowel de gebruiker als de ontwikkelaar die beschikbaar zijn voor iOS en Android nagaan. Om antwoord te formuleren op de tweede onderzoeksvraag zal dit onderzoek nagaan welke richtlijnen we kunnen vormen om ontwikkelaars in staat te stellen om een toegankelijke app te maken. Deze richtlijnen zullen we onderbrengen in een matrix met een bijhorende schaal. 
Aan de hand van deze richtlijnen zullen er enkele populaire applicaties getest worden op toegankelijkheid. De matrix zal een ontwikkelaar in staat stellen om zijn applicatie te evalueren en eventueel aan te passen naar de bijpassende richtlijnen opgesteld in dit onderzoek. 

Tot slot zal er nagegaan worden wat relevante aanpassingen zijn voor toegankelijkheid die ook het gebruikersgemak verhoogt voor mensen zonder een beperking. Aan de hand van een vragenlijst zal bewezen worden of enkele toegevoegde functionaliteiten voor toegankelijkheid effectief het gebruikersgemak verhoogt.


%---------- Verwachte resultaten ----------------------------------------------
\section{Verwachte resultaten}
\label{sec:verwachte_resultaten}

 Zowel Android als iOS bieden een uitgebreid assortiment aan functionaliteiten voor zowel de gebruiker als de ontwikkelaar. Na het opstellen van de richtlijnen zal blijken dat een groot aantal bestaande apps onvoldoende zullen scoren. Voor de laatste onderzoeksvraag wordt er verwacht dat er vele positieve effecten zijn op gebruikers, vaak zullen ze niet beseffen dat een bepaalde functie ontworpen is voor mensen die nood hebben aan toegankelijkheid.

%---------- Verwachte conclusies ----------------------------------------------
\section{Verwachte conclusies}
\label{sec:verwachte_conclusies}

Wanneer de resultaten van de testen op verschillende apps met onze richtlijnen kloppen mag er geconcludeerd worden dat er onvoldoende aandacht wordt besteed aan het ontwikkelen waarbij men toegankelijkheid in acht neemt. Ondanks de grote hoeveelheid resources heerst er een grote onwetendheid bij ontwikkelaars. Ook het onderwijs spendeert hier onvoldoende aandacht aan.


%------------------------------------------------------------------------------
% Referentielijst
%------------------------------------------------------------------------------
% TODO: de gerefereerde werken moeten in BibTeX-bestand ``voorstel.bib''
% voorkomen. Gebruik JabRef om je bibliografie bij te houden en vergeet niet
% om compatibiliteit met Biber/BibLaTeX aan te zetten (File > Switch to
% BibLaTeX mode)

\phantomsection
\printbibliography[heading=bibintoc]

\end{document}
