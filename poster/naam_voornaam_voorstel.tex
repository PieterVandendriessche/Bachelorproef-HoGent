%==============================================================================
% Sjabloon onderzoeksvoorstel bachelorproef
%==============================================================================
% Gebaseerd op LaTeX-sjabloon ‘Stylish Article’ (zie voorstel.cls)
% Auteur: Jens Buysse, Bert Van Vreckem
%
% Compileren in TeXstudio:
%
% - Zorg dat Biber de bibliografie compileert (en niet Biblatex)
%   Options > Configure > Build > Default Bibliography Tool: "txs:///biber"
% - F5 om te compileren en het resultaat te bekijken.
% - Als de bibliografie niet zichtbaar is, probeer dan F5 - F8 - F5
%   Met F8 compileer je de bibliografie apart.
%
% Als je JabRef gebruikt voor het bijhouden van de bibliografie, zorg dan
% dat je in ``biblatex''-modus opslaat: File > Switch to BibLaTeX mode.

\documentclass{voorstel}

\usepackage{lipsum}

%------------------------------------------------------------------------------
% Metadata over het voorstel
%------------------------------------------------------------------------------

%---------- Titel & auteur ----------------------------------------------------

% TODO: geef werktitel van je eigen voorstel op
\PaperTitle{Titel voorstel}
\PaperType{Onderzoeksvoorstel Bachelorproef 2018-2019} % Type document

% TODO: vul je eigen naam in als auteur, geef ook je emailadres mee!
\Authors{Steven Stevens\textsuperscript{1}} % Authors
\CoPromotor{Piet Pieters\textsuperscript{2} (Bedrijfsnaam)}
\affiliation{\textbf{Contact:}
  \textsuperscript{1} \href{mailto:steven.stevens.u1234@student.hogent.be}{steven.stevens.u1234@student.hogent.be};
  \textsuperscript{2} \href{mailto:piet.pieters@acme.be}{piet.pieters@acme.be};
}

%---------- Abstract ----------------------------------------------------------

\Abstract{Hier schrijf je de samenvatting van je voorstel, als een doorlopende tekst van één paragraaf. Wat hier zeker in moet vermeld worden: \textbf{Context} (Waarom is dit werk belangrijk?); \textbf{Nood} (Waarom moet dit onderzocht worden?); \textbf{Taak} (Wat ga je (ongeveer) doen?); \textbf{Object} (Wat staat in dit document geschreven?); \textbf{Resultaat} (Wat verwacht je van je onderzoek?); \textbf{Conclusie} (Wat verwacht je van van de conclusies?); \textbf{Perspectief} (Wat zegt de toekomst voor dit werk?).

Bij de sleutelwoorden geef je het onderzoeksdomein, samen met andere sleutelwoorden die je werk beschrijven.

Vergeet ook niet je co-promotor op te geven.
}

%---------- Onderzoeksdomein en sleutelwoorden --------------------------------
% TODO: Sleutelwoorden:
%
% Het eerste sleutelwoord beschrijft het onderzoeksdomein. Je kan kiezen uit
% deze lijst:
%
% - Mobiele applicatieontwikkeling
% - Webapplicatieontwikkeling
% - Applicatieontwikkeling (andere)
% - Systeembeheer
% - Netwerkbeheer
% - Mainframe
% - E-business
% - Databanken en big data
% - Machineleertechnieken en kunstmatige intelligentie
% - Andere (specifieer)
%
% De andere sleutelwoorden zijn vrij te kiezen

\Keywords{Onderzoeksdomein. Keyword1 --- Keyword2 --- Keyword3} % Keywords
\newcommand{\keywordname}{Sleutelwoorden} % Defines the keywords heading name

%---------- Titel, inhoud -----------------------------------------------------

\begin{document}

\flushbottom % Makes all text pages the same height
\maketitle % Print the title and abstract box
\tableofcontents % Print the contents section
\thispagestyle{empty} % Removes page numbering from the first page

%------------------------------------------------------------------------------
% Hoofdtekst
%------------------------------------------------------------------------------

% De hoofdtekst van het voorstel zit in een apart bestand, zodat het makkelijk
% kan opgenomen worden in de bijlagen van de bachelorproef zelf.
%---------- Inleiding ---------------------------------------------------------

\section{Introductie} % The \section*{} command stops section numbering
\label{sec:introductie}

Smartphones hebben een grote impact op ons hedendaagse leven. Het heeft de sociale en economische betrokkenheid van een individu veranderd.  Toch is het gebruik van een smartphone gelimiteerd voor mensen met een beperking
~\autocite{morris2014wireless}. Deze groep dreigt door deze limitatie uitgesloten te worden van bepaalde informatie. De Belgische overheid heeft doelstellingen gemaakt om tegen 22 juni 2021 al hun mobiele applicaties toegankelijk te maken voor mensen met een beperking. Hierbij volgen ze de richtlijnen op van de Europese Unie voor toegankelijkheid van digitale informatie~\autocite{Knacktoegankelijkheid2018}.
\\~\\
Hedendaags is toegankelijkheid een onderwerp die we niet kunnen vermijden. Toch is er binnen de software-development een gebrek aan duidelijke richtlijnen. Juist daarom wordt dit onderwerp vaak overgeslagen bij het maken van mobiele applicaties. Ook de exponentiële groei van innovatie bij smartphones beperkt het opleggen van duidelijke richtlijnen~\autocite{diaz2014accessibility}. \\~\\ Er zijn  tal van functionaliteiten die beschikbaar worden gesteld op mobiele platformen om deze toegankelijker te maken. Ontwikkelaars worden daarbij voorzien van uitgebreide software bibliotheken. Deze functionaliteiten en bibliotheken verschillen per platform, waardoor het aanbod aan voorzieningen voor bepaalde beperkingen kan verschillen. In dit onderzoek gaan we voor de platformen Android en IOS het volgende nagaan:


% TODO: Deftige formulering van onderzoeksvragen


\begin{itemize}
    \setlength\itemsep{0.5 em}
    \item Wat zijn de overeenkomsten en verschillen tussen de platformen IOS en Android inzake toegankelijkheid?
  \item Hoe kan men aanpassingen doen aan een mobiele applicatie met betrekking tot toegankelijkheid?
  \item Hebben aanpassingen voor toegankelijkheid in mobiele applicaties een effect op het gebruik ervan door mensen met een beperking?
  
\end{itemize}

%---------- Stand van zaken ---------------------------------------------------

\section{State-of-the-art}
\label{sec:state-of-the-art}

Onderzoek naar toegankelijkheid in mobiele applicaties heeft er in het verleden al plaatsgevonden. Door de snelle innovatie en groei van mobiele platformen lijken vele onderzoeken gedateerd. 

Vaak zijn deze onderzoeken ook voor een specifieke beperking. Zo omschreef het onderzoek van \citeauthor{leporini2012interacting} wat de impact van de VoiceOver functie in IOS was op mensen met een visuele beperking. Er werd in dit onderzoek geconcludeerd dat ondaks de krachtige mogelijkheden van VoiceOver de applicaties niet voldoende de interactieve elementen beschrijven zodat VoiceOver deze beschrijving kan voorlezen~\autocite{leporini2012interacting}. 

Het onderzoek van \citeauthor{diaz2014accessibility} focust zich op de oudste generatie, vaak hebben zij te kampen met een lichamelijke achteruitgang. Ook deze groep heeft daardoor nood aan toegankelijkheid in mobiele applicaties. Gedurende dit onderzoek heeft men onderzoek gedaan om duidelijke richtlijnen te maken voor toegankelijkheid bij ouderen. Ze concluderen dat elke applicatie zou toegankelijk moeten zijn om sociale uitsluiting te voorkomen~\autocite{diaz2014accessibility}. 
\\~\\
Een vergelijkende studie tussen Android en IOS over de beschikbare functionaliteiten werd door \citeauthor{10.1007/978-3-319-07638-6_14} uitgevoerd. Deze studie heeft per type beperking een opsomming gemaakt van enkele functionaliteiten die er per platform beschikbaar zijn. De onderzoeker vermelde dat er in de toekomst nieuwe functionaliteiten zullen zijn die nog betere toegankelijkheid zal verzorgen voor mensen met een beperking bij het gebruik van mobiele applicaties~\autocite{10.1007/978-3-319-07638-6_14}. Dit onderzoek is gepubliceerd in 2014, sinds dien zijn er tal van nieuwe functionaliteiten toegevoegd voor zowel de gebruiker als de ontwikkelaar toegevoegd aan beide platformen.
\\~\\
Het onderzoek zal de focus niet enkel op de functionaliteiten beschikbaar gesteld voor de gebruiker leggen maar ook op de functionaliteiten die de ontwikkelaar kan gebruiken om zijn mobiele applicatie meer toegankelijk te maken. Ook zal nagegaan worden in dit onderzoek of de richtlijnen die geformuleerd worden een effect hebben op het gebruik door mensen met een beperking.
% TODO: state-of-art
%Hier beschrijf je de \emph{state-of-the-art} rondom je gekozen onderzoeksdomein. Dit kan bijvoorbeeld een literatuurstudie zijn. Je mag de titel van deze sectie ook aanpassen (literatuurstudie, stand van zaken, enz.). Zijn er al gelijkaardige onderzoeken gevoerd? Wat concluderen ze? Wat is het verschil met jouw onderzoek? Wat is de relevantie met jouw onderzoek?

%Verwijs bij elke introductie van een term of bewering over het domein naar de vakliteratuur, bijvoorbeeld~\autocite{Doll1954}! Denk zeker goed na welke werken je refereert en waarom.

% Voor literatuurverwijzingen zijn er twee belangrijke commando's:
% \autocite{KEY} => (Auteur, jaartal) Gebruik dit als de naam van de auteur
%   geen onderdeel is van de zin.
% \textcite{KEY} => Auteur (jaartal)  Gebruik dit als de auteursnaam wel een
%   functie heeft in de zin (bv. ``Uit onderzoek door Doll & Hill (1954) bleek
%   ...'')

%Je mag gerust gebruik maken van subsecties in dit onderdeel.

%---------- Methodologie ------------------------------------------------------
\section{Methodologie}
\label{sec:methodologie}

Voor ik de eerste onderzoeksvraag kan beantwoorden zal eerst een literatuurstudie naar de verschillende beperkingen plaatsvinden. Daarna volgt een onderzoek per platform wat de \\~\\

% TODO: methadologie
Hier beschrijf je hoe je van plan bent het onderzoek te voeren. Welke onderzoekstechniek ga je toepassen om elk van je onderzoeksvragen te beantwoorden? Gebruik je hiervoor experimenten, vragenlijsten, simulaties? Je beschrijft ook al welke tools je denkt hiervoor te gebruiken of te ontwikkelen.

%---------- Verwachte resultaten ----------------------------------------------
\section{Verwachte resultaten}
\label{sec:verwachte_resultaten}

Hier beschrijf je welke resultaten je verwacht. Als je metingen en simulaties uitvoert, kan je hier al mock-ups maken van de grafieken samen met de verwachte conclusies. Benoem zeker al je assen en de stukken van de grafiek die je gaat gebruiken. Dit zorgt ervoor dat je concreet weet hoe je je data gaat moeten structureren.

%---------- Verwachte conclusies ----------------------------------------------
\section{Verwachte conclusies}
\label{sec:verwachte_conclusies}

Hier beschrijf je wat je verwacht uit je onderzoek, met de motivatie waarom. Het is \textbf{niet} erg indien uit je onderzoek andere resultaten en conclusies vloeien dan dat je hier beschrijft: het is dan juist interessant om te onderzoeken waarom jouw hypothesen niet overeenkomen met de resultaten.



%------------------------------------------------------------------------------
% Referentielijst
%------------------------------------------------------------------------------
% TODO: de gerefereerde werken moeten in BibTeX-bestand ``voorstel.bib''
% voorkomen. Gebruik JabRef om je bibliografie bij te houden en vergeet niet
% om compatibiliteit met Biber/BibLaTeX aan te zetten (File > Switch to
% BibLaTeX mode)

\phantomsection
\printbibliography[heading=bibintoc]

\end{document}
